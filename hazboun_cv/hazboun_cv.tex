\documentclass[11pt,letterpaper,sans,unicode]{moderncv}
\usepackage{xpatch}
\usepackage{graphicx}
\usepackage{longtable}
\usepackage{multirow}
%\usepackage{enumitem}
\usepackage{relsize}
\usepackage{textcomp}
\usepackage{units}
\usepackage{lineno}
\usepackage{rotating}
\usepackage{amssymb}
\usepackage{amsmath}
\usepackage[utf8]{inputenc}
\usepackage{longtable}
\usepackage{xcolor}
\usepackage[gen]{eurosym}
\usepackage{comment}
\usepackage{etaremune}

%\usepackage[T1]{fontenc}
%\usepackage{hyperref}
%\usepackage[latin9]{inputenc}


\moderncvstyle{banking}
\moderncvcolor{blue}

\renewcommand{\familydefault}{\sfdefault}
\usepackage[top=0.7in,bottom=0.7in,left=1in,right=1in,bindingoffset=0cm]{geometry}
\setlength{\hintscolumnwidth}{3cm}
\usepackage{enumitem}
\setlist{nolistsep}
\usepackage{lastpage}
\usepackage{mathabx}
\cfoot{{\color{gray} Page \thepage\ of \pageref{LastPage}}}

\def\baas{{Bulletin of AAS}}               % Bulletin of the AAS
\def\prd{{Phys. Rev.} D}
\def\prl{{Phys.Rev.} Lett}
\def\ajp{{Amer. J. Phys}}
\def\apjl{{Astrophys. J.} Lett}
\def\apjs{{Astrophys. J.} Supp}
\def\apj{{Astrophys. J.}}
\def\cqg{{Class. Quantum Grav.}}
\def\aaps{{A\&AS}}
\def\pasj{{PASJ}}
\def\mnras{{MNRAS}}
\def\aapr{{A\&ARv}}
\def\aap{{A\&A}}
\def\grg{{Gen. Rel.\&Grav.}}
\def\na{{New Astronomy}}
\def\ptp{{Progress of Theoretical Physics}}
\def\araa{{ARA\&A}}
\def\ssr{{Space Sci. Rev.}}
\def\jpcs{{J. Phys. Conf. Ser.}}
\def\npb{{Nucl. Phys. B}}
\def\clrpaawp{{Can. Lon. Ran. Plan A\&A}}


\newcommand{\cvreference}[9]{%
  \textbf{#1}\newline% Name
  \ifthenelse{\equal{#2}{}}{}{\addresssymbol~#2\newline}%
  \ifthenelse{\equal{#3}{}}{}{#3\newline}%
  \ifthenelse{\equal{#4}{}}{}{#4\newline}%
  \ifthenelse{\equal{#5}{}}{}{#5\newline}%
  \ifthenelse{\equal{#6}{}}{}{#6\newline}%
  \ifthenelse{\equal{#7}{}}{}{#7\newline}%
  \ifthenelse{\equal{#8}{}}{}{\emailsymbol~{\footnotesize \texttt{\emaillink[]{#8}}}\newline}
  \ifthenelse{\equal{#9}{}}{}{\phonesymbol~#9}
  }


%%%%%% My Macros %%%%%
\newcommand{\editem}[7]{\cventry{#1}{\hspace{4mm}#2 \vspace*{1mm}}{#3}{#4}{}{\hspace{4mm}\textbf{Advisor:} #5}{\hspace{4mm}\textbf{#6:} \textit{#7} }}
%School City/State, School, Degree, Degree Date, Advisor, title type, title

\newcommand{\bsitem}[4]{\cventry{#1}{\hspace{4mm}#2 \vspace*{1mm}}{#3}{#4}{}{}{}}
%School City/State, School, Degree, Degree Date

\newcommand{\pubitem}[4]{\item \cvitem{}{\textit{#1}.\\ {#2} \\{\color{color1} \href{#3}{\color{color1} #4}} }\vspace{-0.3cm}} %title, authors, href, journal %\textcolor{red}{$\bullet$} $141$ citations.

\newcommand{\software}[3]{\cventry{}{\bbullet #2}{\hspace{2mm}\texttt{#1}}{}{}{\bbullet \color{color1} \url{#3}}} %Name, Description, URL
%\vspace{-0.1cm}

\newcommand{\talkitem}[3]{\item \textit{#1}\\{#2}, #3} %Title, Meeting, Date
\newcommand{\invtalkitem}[3]{\item \textbf{#1}, \hfill{#2} \\ \textit{\color{color1}``#3''} \vspace{-0.1cm}} %Place, Title, Date

%\newcommand{\invtalkitem}[3]{\cvitemwithcomment{}{\item #1 \color{color1}\textit{``#2''}}{#3} \vspace{-0.1cm}}
\newcommand{\outtalkitem}[4]{\cvitemwithcomment{}{\bbullet #1, #2 \color{color1}\textit{``#3''}}{#4} \vspace{-0.1cm}} %What, Where, Title, Year
\newcommand{\studentitem}[4]{\item \textit{#1}, {#2} \hfill{#3} \\ \textit{``#4''} } %Name, School, Date, Title\vspace{-0.1cm}
\newcommand{\confitem}[2]{\item #1 \hfill #2} %Meeting, Date

\newcommand{\blucirc}{{\color{color1} $\circ\;\;$}}
\newcommand{\bbullet}{\hspace{0.4cm}\blucirc}

%----------------------------------------------------------------------------------------
%   NAME AND CONTACT INFORMATION SECTION
%----------------------------------------------------------------------------------------

\firstname{\huge{Jeremy~G.}} % Your first name
\familyname{\huge{Baier}} % Your last name

% All information in this block is optional, comment out any lines you don't need
\title{\huge{Curriculum Vitae}}
\address{Department of Physics, Oregon State University}{401 Weniger Hall Corvallis, OR 97331-6507, USA}
%\phone[mobile]{+1~(801)~440~2156}
\phone{+1~(504)~913~4014}
\email{baierj@oregonstate.edu}
\homepage{jeremy-baier.github.io}
\social[github]{jeremy-baier}
\social[twitter]{jeremygbaier}
\social[linkedin][linkedin.com/in/]{jeremygbaier}

\lhead{{\color{gray} \fontsize{10}{22}\mdseries\upshape{Jeremy~G.~Baier}}}
\rhead{{\color{gray} \fontsize{10}{10}\mdseries\upshape{baierj@oregonstate.edu}}}

%\definecolor{color1}{HTML}{1f77b4}
%\hypersetup{urlcolor=color1,linkcolor=color1}

\begin{document}
\makecvtitle

\vspace{-13mm}
%----------------------------------------------------------------------------------------
%   WORK EXPERIENCE SECTION
%----------------------------------------------------------------------------------------

\section{Professional Experience}

\cventry{September 2022--Present}{\hspace{0.4cm}{\blucirc}Graduate Teaching Assistant}
{\textsc{Oregon State University}}
{Corvallis, OR}{}{} %\vspace{-6mm}
%

\cventry{August 2019--May 2022}{\hspace{0.4cm}{\blucirc}Researcher at Kenyon LIGO Lab}
{\textsc{Kenyon College}}
{Gambier, OH}{}{} %\vspace{-6mm}
%

%----------------------------------------------------------------------------------------
%   EDUCATION SECTION
%----------------------------------------------------------------------------------------
%\vspace{-3mm}
\section{Education}
%School City/State, School, Degree, Degree Date, Advisor, title type, title

\editem{Corvallis, Oregon}{Oregon State University}{PhD in Physics (in progress)}{Sept 2022 - present}{Dr.~Jeffrey~S.~Hazboun}{}{}
\vspace{2mm}

\editem{Gambier, Ohio}{Kenyon College}{BA Mathematics & Physics}{May 2022}{Dr.~Leslie Wade}{Capstone Title}{Constraining the Neutron Star Equation of State through binary neutron star mergers}
%\vspace{2mm}

%\bsitem{Syracuse, New York}{State University of New York, College of Environmental Science and Forestry}{BS Biology}{December 1999}

%----------------------------------------------------------------------------------------
%   RESEARCH SECTION
%----------------------------------------------------------------------------------------
%\vspace{-4mm}
%\section{Research Interests}
%\cvitem{Primary interests}{gravitational-wave astronomy~$\bullet$~theoretical astrophysics~$\bullet$~massive black-hole binaries~$\bullet$~stellar-mass compact objects~$\bullet$~pulsar timing~$\bullet$~statistical inference}
%\cvitem{Secondary interests}{galaxy formation and evolution~$\bullet$~cosmology~$\bullet$~pulsar physics and demographics~$\bullet$~ionized interstellar medium}
%\cvitem{Specific interests}{Bayesian hierarchical modeling~$\bullet$~pulsar-timing data-analysis for nanohertz gravitational-wave searches~$\bullet$~compact-binary demographics and population inference~$\bullet$~pulsar-timing noise characterization and mitigation~$\bullet$~waveform modeling for supermassive black-hole binary searches~$\bullet$~modeling final-parsec dynamics of supermassive black-hole binaries~$\bullet$~stochastic signal analysis strategies}

%----------------------------------------------------------------------------------------
%   GRANTS & FUNDING
%----------------------------------------------------------------------------------------

%\section{Grants, Funding \& Awards}

%\cvitemwithcomment{}{\textbf{Proposal (PI): ``Pulsar Timing Array GW Signal Analysis Using Big Data Techniques''}}{2019}
%\begin{tabular}{rcl}
%&\hspace{0.4cm} &{\color{color1} $\circ\;\;$}PI Jeffrey Hazboun, a submission to Amazon Web Services Machine Learning Research Awards. \\
%&\hspace{0.4cm} &  {\color{color1} $\circ\;\;$}Total award: $\$50,000$ in AWS Promotional Credits.\\
%\end{tabular} \\

%----------------------------------------------------------------------------------------
%  OBSERVING PROPOSALS
%----------------------------------------------------------------------------------------
%\vspace{-4mm}
%\section{Observing Proposals}
%{\small
%\cvitemwithcomment{}{Co-I: \textbf{``Monitoring pulse-shape changes in the IPTA pulsar sample"}}{May 2021}
%\begin{tabular}{rcl}
%&\hspace{0.4cm} &{\color{color1} $\circ\;\;$}Giant Metrewave Radio Telescope, Target of Opportunity proposal  \\
%&\hspace{0.4cm} &{\color{color1} $\circ\;\;$}Status: awarded $10.0$ hours
%\end{tabular} \\

%\cvitemwithcomment{}{Co-I: \textbf{``Tracking Rapid and Unexpected Pulse Shape Changes in the MSP~J1713+0747"}}{May 2021}
%\begin{tabular}{rcl}
%&\hspace{0.4cm} &{\color{color1} $\circ\;\;$}Very Large Array, Director's discretionary time proposal VLA/$21A-426$ \\
%&\hspace{0.4cm} &{\color{color1} $\circ\;\;$}Status: awarded $14.0$ hours
%\end{tabular} \\

%\cvitemwithcomment{}{Co-I: \textbf{``Monitoring pulse shape changes in the International Pulsar Timing Array"}}{June 2021}
%\begin{tabular}{rcl}
%&\hspace{0.4cm} &{\color{color1} $\circ\;\;$}Parkes Observatory, Non A-priori Assignable Proposal \\
%&\hspace{0.4cm} &{\color{color1} $\circ\;\;$}Status: awarded $10.0$ hours
%\end{tabular} \\

%\cvitemwithcomment{}{Co-I: \textbf{``High Cadence Observations of MSPs for Gravitational Wave Detection"}}{March 2020}
%\begin{tabular}{rcl}
%&\hspace{0.4cm} &{\color{color1} $\circ\;\;$}Arecibo Radio Telescope, proposal P$2945$ \\
%&\hspace{0.4cm} &{\color{color1} $\circ\;\;$}Status: awarded $32.5$ hours
%\end{tabular} \\

%\cvitemwithcomment{}{Co-I: \textbf{``High Time Resolution Observations of a Bright Millisecond Pulsar"}}{November 2018}
%\begin{tabular}{rcl}
%&\hspace{0.4cm} &{\color{color1} $\circ\;\;$}Greenbank Telescope, Project ID GBT$18B-355$ \\
%&\hspace{0.4cm} &{\color{color1} $\circ\;\;$}Status: awarded $5$ hours
%\end{tabular}
%}

%\vspace{-10mm}
%----------------------------------------------------------------------------------------
%  FULL PUBLICATION LIST
%----------------------------------------------------------------------------------------

%\section{Publications}

%\cvitem{}
%{
%\begin{tabular}{rcl}
%\textbf{Counts}: &\hspace{0.3cm} &{\textbf{44} papers published in major peer-reviewed journals} \\
%& &{(out of which \textbf{12} are first-authored papers, and \textbf{4} papers were covered by press releases).}
%\end{tabular}
%}

%$\bullet$ \textbf{Total number of citations:} 2650 (using Google Scholar), \textbf{h-index:} 27, \textbf{i10-index:} 34

%$\bullet$ \textbf{Metrics} available at {\color{color1} \href{https://inspirehep.net/literature?sort=mostrecent&size=25&page=1&q=a%20J.S.Hazboun.1&ui-citation-summary=true}{InspireHEP} or \href{https://scholar.google.com/citations?user=CbNY_MYAAAAJ&hl=en}{Google Scholar}}.

%\subsection{Submitted}

%\begin{etaremune}[leftmargin=8mm]
%\pubitem{{Inferring Mbh-Mbulge Evolution from the Gravitational Wave Background}}
         {Cayenne~{Matt}, [...], \textbf{J.~G.~{Baier}}, et al. [107 Authors]}
         {https://doi.org/10.48550/arXiv.2508.18126}
         {{Arxiv:}2508.18126}

\pubitem{{The NANOGrav 15 yr Data Set: Targeted Searches for Supermassive Black Hole Binaries}}
         {Nikita~{Agarwal}, [...], \textbf{J.~G.~{Baier}}, et al. [119 Authors]}
         {https://doi.org/10.48550/arXiv.2508.16534}
         {{Arxiv:}2508.16534}

\pubitem{{Galaxy Tomography with the Gravitational Wave Background from Supermassive Black Hole Binaries}}
         {Yifan~{Chen}, [...], \textbf{J.~G.~{Baier}}, et al. [110 Authors]}
         {https://doi.org/10.48550/arXiv.2411.05906}
         {{Arxiv:}2411.05906}


%\end{etaremune}

%\subsection{Accepted}

%\begin{etaremune}[leftmargin=8mm]
%\input{accept_entries}
%\end{etaremune}

%\subsection{Published}

%\begin{etaremune}[leftmargin=8mm]
%\pubitem{{Disentangling Multiple Stochastic Gravitational Wave Background Sources in PTA Datasets}}
         {Andrew R.~{Kaiser}, [...], \textbf{J.~S.~{Hazboun}}, et al. [10 Authors]}
         {https://doi.org/{10.3847/1538-4357/ac86cc}}
         {{The Astrophysical Journal}, \textbf{{938}}, {2}, (2022)}

\pubitem{{Bayesian Solar Wind Modeling with Pulsar Timing Arrays}}
         {\textbf{J.~S.~{Hazboun}}, et al. [30 Authors]}
         {https://doi.org/10.3847/1538-4357/ac5829}
         {{The Astrophysical Journal}, \textbf{929}, 1, (2022)}

\pubitem{{A Detection of Red Noise in PSR J1824$-$2452A and\\Projections for PSR B1937+21 using NICER X-ray Timing Data}}
         {\textbf{J.~S.~{Hazboun}}, et al. [20 Authors]}
         {https://doi.org/10.3847/1538-4357/ac54ae}
         {{The Astrophysical Journal}, \textbf{928}, 1, (2022)}

\pubitem{{The International Pulsar Timing Array second data release:\\ Search for an isotropic Gravitational Wave Background}}
         {J.~{Antoniadis}, [...], \textbf{J.~S.~{Hazboun}}, et al. [70 Authors]}
         {https://doi.org/10.1093/mnras/stab3418}
         {{Monthly Notices of the Royal Astronomical Society}, \textbf{510}, 4, (2022)}

\pubitem{The NANOGrav 12.5-year data set: Search for Non-Einsteinian Polarization Modes\\in the Gravitational-Wave Background}
         {Z.~{Arzoumanian}, [...], \textbf{J.~S.~{Hazboun}}, et al. [71 Authors]}
         {https://doi.org/10.3847/2041-8213/ac401c}
         {{The Astrophysical Journal Letters}, \textbf{923}, 2, (2021)}

\pubitem{{Searching For Gravitational Waves From Cosmological Phase Transitions\\With The NANOGrav 12.5-year dataset}}
         {Z.~{Arzoumanian}, [...], \textbf{J.~S.~{Hazboun}}, et al. [65 Authors]}
         {https://doi.org/10.1103/PhysRevLett.127.251302}
         {{Physical Review Letters}, \textbf{127}, 25, (2021)}

\pubitem{{Multimessenger pulsar timing array constraints on supermassive black hole binaries traced by periodic light curves}}
         {Chengcheng~{Xin}, Chiara M.~F.~{Mingarelli}, \textbf{J.~S.~Hazboun}}
         {https://doi.org/10.3847/1538-4357/ac01c5}
         {{The Astrophysical Journal}, \textbf{915}, 2, (2021)}

\pubitem{{The NANOGrav 11yr Data Set: Limits on Supermassive Black Hole Binaries in Galaxies within 500Mpc}}
         {Z.~{Arzoumanian}, [...], \textbf{J.~S.~{Hazboun}}, et al. [57 Authors]}
         {https://doi.org/10.3847/1538-4357/abfcd3}
         {{The Astrophysical Journal}, \textbf{914}, 2, (2021)}

\pubitem{{Astrophysics Milestones For Pulsar Timing Array Gravitational Wave Detection}}
         {N.~S.~{Pol}, [...], \textbf{J.~S.~{Hazboun}}, et al. [51 Authors]}
         {https://doi.org/10.3847/2041-8213/abf2c9}
         {{The Astrophysical Journal Letters}, \textbf{911}, 2, (2021)}

\pubitem{{Precision Timing of PSR J0437-4715 with the IAR Observatory and Implications for Low-Frequency Gravitational Wave Source Sensitivity}}
         {M.~T.~{Lam}, \textbf{J.~S.~Hazboun}}
         {https://doi.org/10.3847/1538-4357/abeb64}
         {{The Astrophysical Journal}, \textbf{911}, 2, (2021)}

\pubitem{A Study in Frequency-Dependent Effects on Precision Pulsar Timing Parameters with the Pulsar Signal Simulator}
         {B.~J.~Shapiro-Albert, \textbf{J.~S.~Hazboun}, M.~A.~{McLaughlin}, M.~T.~{Lam}}
         {https://doi.org/10.3847/1538-4357/abdc29}
         {{The Astrophysical Journal}, \textbf{909}, 2, (2021)}

\pubitem{{Common-spectrum process versus cross-correlation for gravitational-wave searches using pulsar timing arrays}}
         {J.~D.~{Romano}, \textbf{J.~S.~Hazboun}, X.~{Siemens}, A.~M.~{Archibald}}
         {https://doi.org/10.1103/PhysRevD.103.063027}
         {{Physical Review D}, \textbf{103}, 6, (2021)}

\pubitem{The Pulsar Signal Simulator: A Python package for simulating radio signal data from pulsars}
         {\textbf{J.~S.~{Hazboun}}, et al. [10 Authors]}
         {https://doi.org/10.21105/joss.02757}
         {{Journal of Open Software Science}, \textbf{6}, 58, (2021)}

\pubitem{Model Dependence of Bayesian Gravitational-Wave Background Statistics for Pulsar Timing Arrays}
         {\textbf{J.~S.~Hazboun}, J.~{Simon}, X.~{Siemens}, J.~D.~{Romano}}
         {https://doi.org/10.3847/2041-8213/abca92}
         {{The Astrophysical Journal Letters}, \textbf{905}, 1, (2020)}

\pubitem{{The NANOGrav 12.5-year Data Set: Observations and Narrowband Timing of 47 Millisecond Pulsars}}
         {Md F.~{Alam}, [...], \textbf{J.~S.~{Hazboun}}, et al. [70 Authors]}
         {https://doi.org/10.3847/1538-4365/abc6a0}
         {{The Astrophysical Journal Supplements}, \textbf{252}, 4, (2020)}

\pubitem{{The NANOGrav 12.5-year Data Set: Search For An Isotropic Stochastic Gravitational-Wave Background}}
         {Z.~{Arzoumanian}, [...], \textbf{J.~S.~{Hazboun}}, et al. [61 Authors]}
         {https://doi.org/10.3847/2041-8213/abd401}
         {{The Astrophysical Journal Letters}, \textbf{905}, 2, (2020)}

\pubitem{{Multi-Messenger Gravitational Wave Searches with Pulsar Timing Arrays: \\Application to 3C66B Using the NANOGrav 11-year Data Set}}
         {Z.~{Arzoumanian}, [...], \textbf{J.~S.~{Hazboun}}, et al. [59 Authors]}
         {https://doi.org/10.3847/1538-4357/ababa1}
         {{The Astrophysical Journal}, \textbf{900}, 2, (2020)}

\pubitem{{The NANOGrav 12.5-year Data Set: Wideband Timing of 47 Millisecond Pulsars}}
         {Md F.~{Alam}, [...], \textbf{J.~S.~{Hazboun}}, et al. [70 Authors]}
         {https://doi.org/10.3847/1538-4365/abc6a1}
         {{The Astrophysical Journal Supplements}, \textbf{252}, 1, (2020)}

\pubitem{{Modeling the Uncertainties of Solar System Ephemerides for Robust Gravitational-wave Searches with Pulsar-timing Arrays}}
         {M.~{Vallisneri}, [...], \textbf{J.~S.~{Hazboun}}, et al. [64 Authors]}
         {https://doi.org/10.3847/1538-4357/ab7b67}
         {{The Astrophysical Journal}, \textbf{893}, 2, (2020)}

\pubitem{{The NANOGrav 11 yr Data Set: Evolution of Gravitational-wave Background Statistics}}
         {\textbf{J.~S.~{Hazboun}}, et al. [63 Authors]}
         {https://doi.org/10.3847/1538-4357/ab68db}
         {{The Astrophysical Journal}, \textbf{890}, 2, (2020)}

\pubitem{{The NANOGrav 11 yr Data Set: Limits on Gravitational Wave Memory}}
         {K.~{Aggarwal}, [...], \textbf{J.~S.~{Hazboun}}, et al. [61 Authors]}
         {https://doi.org/10.3847/1538-4357/ab6083}
         {{The Astrophysical Journal}, \textbf{889}, 1, (2020)}

\pubitem{{The International Pulsar Timing Array: second data release}}
         {B.~B.~P.~{Perera}, [...], \textbf{J.~S.~{Hazboun}}, et al. [75 Authors]}
         {https://doi.org/10.1093/mnras/stz2857}
         {{Monthly Notices of the Royal Astronomical Society}, \textbf{490}, 4, (2019)}

\pubitem{{The NANOGrav 11 yr Data Set: \\Limits on Gravitational Waves from Individual Supermassive Black Hole Binaries}}
         {K.~{Aggarwal}, [...], \textbf{J.~S.~{Hazboun}}, et al. [64 Authors]}
         {https://doi.org/10.3847/1538-4357/ab2236}
         {{The Astrophysical Journal}, \textbf{880}, 2, (2019)}

\pubitem{{The astrophysics of nanohertz gravitational waves}}
         {S.~{Burke-Spolaor}, [...], \textbf{J.~S.~{Hazboun}}, et al. [15 Authors]}
         {https://doi.org/10.1007/s00159-019-0115-7}
         {{The Astronomy and Astrophysics Review}, \textbf{27}, 1, (2019)}

\pubitem{{Hasasia: A Python package for Pulsar Timing Array Sensitivity Curves}}
         {\textbf{J.~S.~Hazboun}, J.~D.~{Romano}, T.~L.~{Smith}}
         {https://doi.org/10.21105/joss.01775}
         {{Journal of Open Software Science}, \textbf{4}, 42, (2019)}

\pubitem{{Realistic sensitivity curves for pulsar timing arrays}}
         {\textbf{J.~S.~Hazboun}, J.~D.~{Romano}, T.~L.~{Smith}}
         {https://doi.org/10.1103/PhysRevD.100.104028}
         {{Physical Review D}, \textbf{100}, 10, (2019)}

\pubitem{{An acoustical analogue of a galactic-scale gravitational-wave detector}}
         {M.~T.~{Lam}, J.~D.~{Romano}, J.~S.~{Key}, M.~{Normandin}, \textbf{J.~S.~Hazboun}}
         {https://doi.org/10.1119/1.5050190}
         {{American Journal of Physics}, \textbf{86}, 10, (2018)}

\pubitem{{A Second Chromatic Timing Event of Interstellar Origin toward PSR J1713+0747}}
         {M.~T.~{Lam}, [...], \textbf{J.~S.~{Hazboun}}, et al. [37 Authors]}
         {https://doi.org/10.3847/1538-4357/aac770}
         {{The Astrophysical Journal}, \textbf{861}, 2, (2018)}

\pubitem{{The NANOGrav 11-year Data Set: Pulsar-timing Constraints on the Stochastic Gravitational-wave Background}}
         {Z.~{Arzoumanian}, [...], \textbf{J.~S.~{Hazboun}}, et al. [62 Authors]}
         {https://doi.org/10.3847/1538-4357/aabd3b}
         {{The Astrophysical Journal}, \textbf{859}, 1, (2018)}

\pubitem{{Constructing an explicit AdS/CFT correspondence with Cartan geometry}}
         {\textbf{J.~S.~Hazboun}}
         {https://doi.org/10.1016/j.nuclphysb.2018.02.006}
         {{Nuclear Physics B}, \textbf{929}, pp. 254-265, (2018)}

\pubitem{{Power radiated by a binary system in a de Sitter universe}}
         {B.~{Bonga}, \textbf{J.~S.~Hazboun}}
         {https://doi.org/10.1103/PhysRevD.96.064018}
         {{Physical Review D}, \textbf{96}, 6, (2017)}

\pubitem{{C7 multi-messenger astronomy of GW sources}}
         {M.~{Branchesi}, [...], \textbf{J.~S.~{Hazboun}}, et al. [45 Authors]}
         {https://doi.org/10.1007/s10714-014-1771-6}
         {{General Relativity and Gravitation}, \textbf{46}, 9, (2014)}

\pubitem{{Time and dark matter from the conformal symmetries of Euclidean space}}
         {\textbf{J.~S.~Hazboun}, J.~T.~Wheeler}
         {https://doi.org/10.1088/0264-9381/31/21/215001}
         {{Classical and Quantum Gravity}, \textbf{31}, 21, (2014)}

\pubitem{{A systematic construction of curved phase space: A gravitational gauge theory with symplectic form}}
         {\textbf{J.~S.~Hazboun}, J.~T.~{Wheeler}}
         {https://doi.org/10.1088/1742-6596/360/1/012013}
         {{Journal of Physics: Conference Series}, \textbf{360}, 012013, (2012)}

\pubitem{The Effect of Negative-Energy Shells on the Schwarzschild Black Hole}
         {\textbf{J.~S.~Hazboun}, T.~{Dray}}
         {https://doi.org/10.1007/s10714-009-0916-5}
         {{General Relativity and Gravitation}, \textbf{42}, pp. 1457-1467, (2010)}


%\end{etaremune}

%\subsection{Technical and White Papers}

%\begin{etaremune}[leftmargin=8mm]
%\pubitem{{Heliosphere Meets Interstellar Medium, in a Galactic Context}}
         {Stella Koch~{Ocker}, [...], \textbf{J.~S.~{Hazboun}}, et al. [11 Authors]}
         {https://arxiv.org/abs/2208.11804}
         {{Arxiv:}2208.11804}

\pubitem{{Pulsar Timing Arrays: Gravitational Waves from Supermassive Black Holes and More}}
         {I.~{Stairs}, [...], \textbf{J.~S.~{Hazboun}}, et al. [32 Authors]}
         {https://doi.org/10.5281/zenodo.3756164}
         {{Canadian Long Range Plan for Astronony and Astrophysics White Papers}, \textbf{2020}, pp. 16, (2019)}

\pubitem{{The NANOGrav Program for Gravitational Waves and Fundamental Physics}}
         {S.~{Ransom}, [...], \textbf{J.~S.~{Hazboun}}, et al. [15 Authors]}
         {https://arxiv.org/abs/1908.05356}
         {{Bulletin of American Astronomical Society}, \textbf{51}, pp. 195, (2019)}

\pubitem{{NANOGrav Education and Outreach: Growing a Diverse and Inclusive Collaboration for Low-Frequency Gravitational Wave Astronomy}}
         {Timothy~{Dolch}, [...], \textbf{J.~S.~{Hazboun}}, et al. [27 Authors]}
         {https://arxiv.org/abs/1907.07348}
         {{Bulletin of American Astronomical Society}, \textbf{51}, pp. 254, (2019)}

\pubitem{{The Gravitational View of Massive Black Hole Mergers}}
         {Monica~{Colpi}, [...], \textbf{J.~S.~{Hazboun}}, et al. [19 Authors]}
         {https://arxiv.org/abs/1903.06867}
         {{Bulletin of American Astronomical Society}, \textbf{51}, 3, (2019)}

\pubitem{{Physics Beyond the Standard Model With Pulsar Timing Arrays}}
         {Xavier~{Siemens}, \textbf{J.~S.~{Hazboun}}, et al. [8 Authors]}
         {https://arxiv.org/abs/1907.04960}
         {{Bulletin of American Astronomical Society}, \textbf{51}, 3, (2019)}

\pubitem{{The Second International Pulsar Timing Array Mock Data Challenge}}
         {\textbf{J.~S.~Hazboun}, C.~ M.~F.~{Mingarelli}, K.~J.~{Lee}}
         {https://arxiv.org/abs/1810.10527}
         {{Arxiv:}1810.10527}

\pubitem{{Null-stream pointing with pulsar timing arrays}}
         {\textbf{J.~S.~Hazboun}, S.~L.~{Larson}}
         {https://arxiv.org/abs/1607.03459}
         {{Arxiv:}1607.03459}

\pubitem{{Limiting alternative theories of gravity using gravitational wave observations across the spectrum}}
         {\textbf{J.~S.~Hazboun}, M.~P.~{Marcano}, S.~L.~{Larson}}
         {https://arxiv.org/abs/1311.3153}
         {{Arxiv:}1311.3153}


%\end{etaremune}

%----------------------------------------------------------------------------------------
%   TEACHING SECTION
%----------------------------------------------------------------------------------------
%\vspace{-4mm}
\section{Teaching \& Mentoring}

%\textbf{\textcolor{black}{Lecturing:}}
\subsection{Teaching Positions}
	\renewcommand\labelitemi{\blucirc}
	\begin{itemize}[leftmargin=8mm]
	\setlength\itemsep{1mm}
	\item \textbf{\color{color1} Graduate Teaching Assistant}, Oregon State University, \hfill Fall 2022 - Present
        		\newline  \textit{Descriptive Astronomy }
	\item \textbf{\color{color1} Teaching Assistant}, Hendrix College, \hfill Spring 2020 - Spring 2022
        		\newline  \textit{General Physics I, General Physics II, Fields and Spacetime}
		\newline Graded student work and maintained gradebook.
	\end{itemize}


%\subsection{Postdoctoral scholar mentoring}

%\cvitemwithcomment{}{\hspace{0.4cm}${\color{color1} \circ}\;$ \textit{Dr.~Maria~Charisi}, Vanderbilt University}{Oct 2020--}\vspace{-0.1cm}
%\hspace{0.71cm} Vanderbilt Initiative in Data-intensive Astrophysics (VIDA) Fellow %\vspace{-0.1cm}

%\cvitemwithcomment{}{\hspace{0.4cm}${\color{color1} \circ}\;$ \textit{Dr.~Nihan~Pol}, Vanderbilt University}{Sep 2020--}\vspace{-0.1cm}
%\hspace{0.71cm} Vanderbilt Initiative in Data-intensive Astrophysics (VIDA) Fellow %\vspace{-0.1cm}

\subsection{Undergraduate Student Research Mentoring}
\renewcommand\labelitemi{\blucirc}
\begin{itemize}[leftmargin=8mm]
	\studentitem{Rodney Downer}{Oregon State University}{2023}{LISA multimessenger pipeline}
	%\studentitem{Andrew Kaiser}{West Virginia University}{2018-2020}{Bayesian Non-Linear Timing with Gravitational Wave PTA Software}

\end{itemize}

%----------------------------------------------------------------------------------------
%   OUTREACH SECTION
%----------------------------------------------------------------------------------------
%\vspace{-2mm}
\section{Leadership \& Professional Service}

\subsection{Research leadership}

\cvitemwithcomment{}{\hspace{0.15cm}${\color{color1} \circ}\;$ \textit{\textbf{Lead}}, NANOGrav 15-year Detector Characterization}{Mar 2021--Present} \vspace{-0.1cm}

\cvitemwithcomment{}{\hspace{0.15cm}${\color{color1} \circ}\;$ \textit{\textbf{Co-chair}}, IPTA Gravitational Wave Analysis Group}{Jan 2019--Dec 2021} \vspace{-0.1cm}

\cvitemwithcomment{}{\hspace{0.15cm}${\color{color1} \circ}\;$ \textit{\textbf{Co-chair}}, IGRAV Diversity, Equity \& Inclusion Working Group}{Jan 2019--July 2021} \vspace{-0.1cm}

\cvitemwithcomment{}{\hspace{0.15cm}${\color{color1} \circ}\;$ \textit{\textbf{Co-chair}}, IPTA Data Challenge Group}{Mar 2018--Jan 2022} \vspace{-0.1cm}

\subsection{Reviewer for international journals}
\hspace{0.5mm} \blucirc The Astrophysical Journal  \hfill \blucirc Classical and Quantum Gravity \hfill \hspace{-0.2cm} \blucirc Physical Review D

\hspace{0.5mm} \blucirc General Relativity \& Gravitation  \hfill \blucirc Monthly Notices of the Royal Astronomical Society

\hspace{0.5mm} \blucirc Physical Review Letters  \hfill \blucirc European Journal of Physics

%\vspace{-5mm}

%\subsection{Seminar organization}
%\cvitemwithcomment{}{\hspace{0.4cm}${\color{color1} \circ}\;$ Caltech TAPIR Seminar Series \textit{(executive committee)}}{2019} \vspace{-0.1cm}
%\cvitemwithcomment{}{\hspace{0.4cm}${\color{color1} \circ}\;$ Caltech/JPL Association for Gravitational-Wave Research \textit{(executive committee)}}{2017--2019} \vspace{-0.1cm}
%\cvitemwithcomment{}{\hspace{0.4cm}${\color{color1} \circ}\;$ Caltech TAPIR and LIGO postdoctoral lunch seminar series}{2015--2016} \vspace{-0.1cm}
\begin{comment}
\subsection{Committees}
\cvitemwithcomment{}{\hspace{2mm}${\color{color1} \circ}\;$ NANOGrav chapter of the {\color{color1} \href{https://www.aps.org/programs/innovation/fund/idea.cfm}{APS Inclusion, Diversity, \& Equity Alliance}}}{Jul 2020--} \vspace{-0.1cm}
\end{comment}


\subsection{Professional affiliations}
%
\cvitemwithcomment{}{\hspace{0.4cm}${\color{color1} \circ}\;$ LIGO Scientfic Collaboration, \textit{Member}}{} \vspace{-0.1cm}

\cvitemwithcomment{}{\hspace{0.4cm}${\color{color1} \circ}\;$ North American Nanohertz Observatory for Gravitational-waves (NANOGrav), \textit{Associate member}}{} \vspace{-0.1cm}
\cvitemwithcomment{}{\hspace{0.4cm}${\color{color1} \circ}\;$ International Pulsar Timing Array (IPTA), \textit{Member}}{} \vspace{-0.1cm}
\cvitemwithcomment{}{\hspace{0.4cm}${\color{color1} \circ}\;$ American Physical Society (DGRAV), \textit{Member}}{} \vspace{-0.1cm}
%\vspace{-0.1cm}
\cvitemwithcomment{}{\hspace{0.4cm}${\color{color1} \circ}\;$ American Astronomical Society, \textit{Member}}{} \vspace{-0.1cm}

\section{Software Development}

\subsection{Contributing Developer}
\software{Bilby}{A user-friendly Bayesian inference library.}{https://github.com/lscsoft/bilby}
\software{LALsuite}{LIGO algorithm library.}{https://github.com/lscsoft/lalsuite}
\software{ligo-eos-scripts}{Python scripts for post processing and plotting of equation of state parameters.}
\software{hasasia}{Python package for calculating pulsar timing array sensitivity curves and signal-to-noise ratios.}{https://pypi.org/project/hasasia/}
%\vspace{-0.1cm}

%\vspace{-3mm}
\section{Outreach}

\subsection{Outreach}
\renewcommand\labelenumi{\bfseries\theenumi .}
\small
\outtalkitem{North City Tech Meetup}{}{Searching for Gravitational Waves with a Galactic-Scale Detector}{2021}
\outtalkitem{Gravitational Wave Astronomy}{Eastside Preparatory}{How are gravitational waves detected?}{2020}
\outtalkitem{Science Wednesday Panel Discussion}{King's Live Music}{The Science of Time Travel}{2015}
\outtalkitem{Science Fiction Club Talk}{Hendrix College}{Black Holes and Wormholes}{2015}
\outtalkitem{Science Unwrapped (500 person public lecture)}{Utah State University}{Explore to Conserve}{2013}
\outtalkitem{Conservation Club Talk}{Weber State}{A Scientist's Role in Conservation}{2012}
\outtalkitem{Science Unwrapped}{Swaner Ecocenter}{A Scientist's Role in Modern Exploration}{2012}
\outtalkitem{Cache Valley Stargazers Talk}{Logan, UT}{Black Holes: Ninjas of the Night Sky}{2009}

\subsection{Diversity \& Equity}
\begin{itemize}[leftmargin=8mm]
\item Member, Diversity, Inclusion, Culture \& Equity Committee, Oregon State Physics Department, Sept 2022 --
\item Founding Co-chair, Diversity Equity \& Inclusion Working Group,  {\color{color1} \href{https://indico.ego-gw.it/event/29/sessions/53/attachments/420/685/International_Gravitational-Wave_Outreach_Group_Meeting__Briefing_Document.pdf}{International Gravitational Outreach Group}}, Jul 2018--Sep 2021
\item Member of the NANOGrav chapter of the {\color{color1} \href{https://www.aps.org/programs/innovation/fund/idea.cfm}{APS Inclusion, Diversity, \& Equity Alliance}}
\item Local Organizing Committee, UW, Seattle (2019) {\color{color1} \href{https://www.aps.org/programs/women/cuwip/cuwiphistory.cfm}{Conferences for Undergraduate Women in Physics}}
%\item As an LOC member for the Spring 2018 NANOGrav collaboration meeting at UW Bothell, I coordinated with the Bothell Center for Diversity on a seminar to promote allyship, inclusiveness, and equity within the NANOGrav collaboration.
\end{itemize}

\begin{comment}
\subsection{Press releases}

\cvitemwithcomment{}{\hspace{0.4cm}${\color{color1} \circ}\;$ {\color{color1} \href{https://news.vanderbilt.edu/2020/06/30/to-find-giant-black-holes-start-with-jupiter/}{``\textit{To find giant black holes, start with Jupiter}''}}}{Jun 2020} \vspace{-0.1cm}
\hspace{0.71cm} Collaboration research (Vanderbilt press release)  \\
\cvitemwithcomment{}{\hspace{0.4cm}${\color{color1} \circ}\;$ {\color{color1} \href{https://greenbankobservatory.org/to-find-giant-black-holes-start-with-jupiter/}{``\textit{To find giant black holes, start with Jupiter}''}}}{Jun 2020} \vspace{-0.1cm}
\hspace{0.71cm} Collaboration research (Green Bank Observatory press release)  \\
\cvitemwithcomment{}{\hspace{0.4cm}${\color{color1} \circ}\;$ {\color{color1} \href{https://www.jpl.nasa.gov/news/news.php?feature=6998}{``\textit{Listening for Gravitational Waves Using Pulsars}''}}}{Nov 2017} \vspace{-0.1cm}
\hspace{0.71cm} Collaboration research (JPL press release)  \\
%\cvitemwithcomment{}{\hspace{0.4cm}${\color{color1} \circ}\;$ {\color{color1} \href{https://www.simonsfoundation.org/2017/11/13/gravitational-waves-supermassive-black-hole-merger/}{``\textit{Gravitational Waves from Merging Supermassive Black Holes}}}}{Nov 2017} \vspace{-0.1cm}
%\hspace{0.71cm} {\color{color1} \href{https://www.simonsfoundation.org/2017/11/13/gravitational-waves-supermassive-black-hole-merger/}{\textit{Will Be Spotted within 10 Years, New Study Predicts}''}} \\ \vspace{-0.1cm}
%\hspace{0.55 cm} Collaboration collaboration research (Simons Foundation press release) \vspace{0.1cm}\\
\cvitemwithcomment{}{\hspace{0.4cm}${\color{color1} \circ}\;$ {\color{color1} \href{https://public.nrao.edu/news/pressreleases/2016-nanograv-sbr}{``\textit{Gravitational Wave Search Provides Insights into Galaxy Evolution and Mergers}''}}}{Apr 2016} \vspace{-0.1cm}
\hspace{0.71cm} Collaboration research  (NRAO press release)
\end{comment}
\begin{comment}
\subsection{Media coverage}
\begin{itemize}[leftmargin=8mm]
\item Science Interview for National Geographic November 2009
        			\newline Interviewed for \emph{Phenomena: A science salon hosted by National Geographic}
		about the physics in the movie ``Men Who Stare at Goats''.


\item Research profiled in the popular-science book {\color{color1} \href{https://books.google.com/books/about/Ripples_in_Spacetime.html?id=YicuDwAAQBAJ&hl=en}{``Ripples in Spacetime: Einstein, Gravitational Waves, and the Future of Astronomy''}} by Govert Schilling (forward by Martin Rees), Harvard University Press (2017)
\item Interviewed and quoted by \textit{Science} magazine: {\color{color1} \href{http://science.sciencemag.org/content/351/6278/1124}{``In Search of Spacetime Megawaves''}} by Daniel Clery, Science  11 Mar 2016: Vol. 351, Issue 6278, pp. 1124-1125
\item Quoted, with research featured in {\color{color1} \href{https://www.syfy.com/syfywire/pulsars-black-holes-spacetime-center-of-the-solar-system}{\textit{SyFy}}} (article by Phil Plait), {\color{color1} \href{http://gizmodo.com/we-could-find-even-more-gravitational-waves-soon-with-p-1761021828}{\textit{Gizmodo}}}, {\color{color1} \href{https://www.engadget.com/2016/02/25/pulsars-gravitational-waves-black-holes/}{\textit{Engadget}}}, {\color{color1} \href{http://phys.org/news/2016-02-pulsar-web-low-frequency-gravitational.html}{\textit{Phys.org}}},  {\color{color1} \href{http://www.astronomy.com/news/2016/02/pulsar-web-could-detect-gravitational-waves}{\textit{Astronomy magazine}}}, {\color{color1} \href{http://www.universetoday.com/127562/the-future-of-gravitational-wave-astronomy-enhanced-ligo-pulsar-webs-space-interferometers-and-everything/}{\textit{Universe Today}}}
\item Collaboration research featured in {\color{color1} \href{https://www.sciencedaily.com/releases/2016/04/160405122609.htm}{\textit{Science Daily}}}, {\color{color1} \href{https://astronomynow.com/2016/04/06/gravitational-wave-search-provides-insights-into-galaxy-mergers/}{\textit{Astronomy Now} (online)}}
\end{itemize}
\end{comment}

%----------------------------------------------------------------------------------------
%  FULL PRESENTATIONS
%----------------------------------------------------------------------------------------
%\pagebreak
\section{Full Presentation List}
%\vspace{-0.3cm}
\subsection{Invited talks} %\vspace{-0.6cm}
\renewcommand\labelenumi{\bfseries\theenumi .}

\begin{etaremune}[leftmargin=8mm]
\small
\invtalkitem{University of Michigan Colloquium}{October, 2022}{Lumbering Giants \& Nanohertz Unicorns}
\invtalkitem{Oregon State University Colloquium}{March, 2022}{Lumbering Giants \& Nanohertz Unicorns}
\invtalkitem{State University of New York Oswego Seminar}{February, 2022}{Searching for Lumbering Giants}
\invtalkitem{Gravitational Wave Physics and Astronomy Workshop Plenary}{December, 2021}{Current Status of Pulsar Timing Array Gravitational Wave Astronomy}
\invtalkitem{University of South Carolina Colloquium}{December, 2021}{The Search for Lumbering Giants}
\invtalkitem{Idaho State University Colloquium}{October, 2021}{The Search for Lumbering Giants}
\invtalkitem{Gravitational Wave Astronomy Northwest}{July, 2021}{Pulsar Timing Array Gravitational Wave Astronomy Update}
\invtalkitem{Los Alamos National Lab}{April, 2021}{Searching for Nanohertz Gravitational Waves with a Galactic-Scale Detector}
\invtalkitem{University of Missouri}{March, 2021}{Searching for Nanohertz Gravitational Waves with a Galactic-Scale Detector}
\invtalkitem{Univ. of Wisc. Milwaukee, Center for Grav., Cosmo. and Astroph. Seminar}{February, 2021}{Doubling Down on Single Source Sensitivity}
\invtalkitem{CERN Theory Colloquium}{January, 2021}{Highlights from the Search for Gravitational Waves in NANOGrav Datasets}
\invtalkitem{Gravitational Wave Astronomy Northwest}{June, 2020}{Update on the search for gravitational waves in NANOGrav and IPTA datasets}
\invtalkitem{LIGO Hanford Seminar}{March, 2020}{The Search for Lumbering Giants}
\invtalkitem{American Astronomical Society 235th Meeting, NANOGrav Special Session}{January, 2020}{Highlights from the search for gravitational waves in NANOGrav datasets}
\invtalkitem{Montana State University, Physics Colloquium}{November, 2019}{Exploring the discovery space of pulsar timing arrays with realistic sensitivity curves}
\invtalkitem{Whitman College, Physics Colloquium}{October, 2019}{The Search for Lumbering Giants:\\ Observing the Nanohertz Gravitational-Wave Sky with Pulsar Timing Arrays}
\invtalkitem{22nd International Conference on General Relativity and Gravitation (GR22)\\ \& 13th Edoardo Amaldi Conference on Gravitational Waves (Amaldi13)}{July, 2019}{Education and Public Outreach Efforts by Pulsar Timing Array Collaborations}
\invtalkitem{Northwest APS Meeting}{May, 2019}{The Search for Lumbering Giants:\\Observing the Nanohertz Gravitational-Wave Sky with Pulsar Timing Arrays}
\invtalkitem{Gravitational Wave Physics and Astronomy Workshop}{December, 2019}{Current Status of Pulsar Timing Array Gravitational Wave Astronomy}
\invtalkitem{University of Washington Bothell Physical Sciences Division Seminar}{December, 2018}{Observing the Nanohertz Gravitational-Wave Sky with Pulsar Timing Arrays}
\invtalkitem{University of Washington Seattle AstroLunch Talk}{February, 2018}{A Galactic Scale Gravitational Wave Detector: The NANOGrav 11yr Limits}

\invtalkitem{University of Washington Bothell Physical Sciences Division Seminar Seminar}{November, 2017}{The NANOGrav Pulsar Timing Array:\\Using simulations to characterize our galactic gravitational wave detector.}
\invtalkitem{University of Texas Rio Grande Valley Arecibo Remote Command Center Meeting}{February, 2017}{Simulating Pulsar Signals for Noise Characterization of PTAs}
\invtalkitem{University of Arkansas Physics Colloquium}{February, 2016}{Gravitational Wave Astronomy in the 2nd Century of GR}
\invtalkitem{Western Washington University Physics Colloquium}{May, 2015}{A New Window into the Cosmos}
\invtalkitem{Brigham Young University Physics Theory Seminar}{February, 2015}{Gravitational Gauge Theory and the Dark Cosmological Constituents}
\invtalkitem{Georgia Tech Center for Relativistic Astrophysics, Departmental Colloquium}{March, 2013}{Biconformal Space \& Testing Alternative Theories of Gravity using Multi-Messenger Astronomy}
\invtalkitem{Utah State University Physics Colloquium}{February, 2013}{Best Practices for the Online Classroom}
\invtalkitem{Utah State University Physics Colloquium}{September 2010}{Curved Phase Space from conformal symmetry}
\invtalkitem{Oregon State Physics Colloquium}{March 2009}{Spherical Shells in a Schwarzschild Background}

\end{etaremune}
%\vspace{-0.6cm}
\subsection{Contributed presentations} %\vspace{-0.6cm}

\renewcommand\labelenumi{\bfseries\theenumi .}
\begin{etaremune}[leftmargin=8mm]
\small
\talkitem{NANOGrav 15-year Detector Characterization Analysis}{NANOGrav Fall Meeting}{Milwaukee, October, 2022}
\talkitem{Noise Budget for the NANOGrav Pulsar Timing Array}{American Physical Society April Meeting }{New York City, April, 2022}
\talkitem{NANOGrav 15-year Dataset Noise Update}{NANOGrav Spring Meeting}{New York City, March, 2022}
\talkitem{Full PTA Advanced Noise Modeling Update}{NANOGrav Fall Meeting}{Virtual/Nashville TN, October, 2021}
\talkitem{Comparing Single-Source Statistics for PTA Observing Strategies}{American Physical Society April Meeting}{Virtual, April, 2021}
\talkitem{Doubling Down on Single Source Sensitivity}{American Astronomical Society Meeting}{Virtual, January, 2021}
\talkitem{Model Dependence of Bayesian Gravitational Wave Background Statistics in PTAs}{International Pulsar Timing Array Meeting}{Virtual, September, 2020}
\talkitem{Predicting NANOGrav's Sensitivity into the future with \texttt{hasasia}}{American Physical Society April Meeting}{Virtual, April, 2020}
\talkitem{Exploring the Nanohertz Gravitational-Wave Discovery Space with Sensitivity Curves and \texttt{hasasia}}{American Astronomical Society Meeting}{Honolulu, HI, January, 2020}
\talkitem{Gravitational Wave Astronomy with the NANOGrav Pulsar Timing Array}{Texas Symposium on Relativistic Astrophysics}{Portsmouth, UK, December, 2019}
\talkitem{Modeling Astrophysical Noise Sources in PTAs}{Fall NANOGrav Meeting}{Ithaca, NY, October, 2019}
\talkitem{Realistic Pulsar Timing Array Sensitivity Curves}{GR22/Amaldi13}{Valencia, Spain, July, 2019}
\talkitem{Pulsar Timing Array Sensitivity Curves}{American Physical Society April Meeting}{Denver, Colorado, April, 2019}
\talkitem{Characterizing the Sensitivity of the NANOGrav 11-year Data Set}{Spring NANOGrav Meeting} Bothell, Washington, March, 2019
\talkitem{Bayesian Monitoring of Solar Electron Density Using NANOGrav Data sets}{American Astronomical Society Meeting}{Seattle, Washington, January, 2019}
\talkitem{Bayesian Monitoring of the Solar Wind with Pulsar Timing Arrays}{AstroNWxSW}{Vancouver, British Columbia, November, 2018}
\talkitem{Spurious Gravitational Wave Detections in the NANOGrav 11 Year Data Set}{Fall NANOGrav Meeting}{Green Bank, West Virginia, October, 2018}
\talkitem{The International Pulsar Timing Array Mock Data Challenge}{LISA Symposium} Chicago, Illinois, July, 2018
\talkitem{Evolution of the Detection Statistics in the NANOGrav Dataset}{International Pulsar Timing Array Meeting}{Albuquerque, New Mexico, June, 2018}
\talkitem{Noise Evolution in the NANOGrav 11 Year Data Set}{Northwest Section APS Meeting}{Tacoma, Washington, June, 2018}
\talkitem{Publishing a Gravitational Wave Stochastic Background Analysis}{Python in Astronomy}{New York, New York, May, 2018}
\talkitem{Slicing the NANOGrav 11 Year Data Set}{American Physical Society April Meeting}{Columbus, Ohio, April, 2018}
\talkitem{Evolution of the NANOGrav 11 Year Data Set}{Spring NANOGrav Meeting}{Charlottesville, Virginia, March, 2018}
\talkitem{Slicing the NANOGrav 11 Year Data Set}{Fall NANOGrav Meeting} Easton, Pennsylvania, October, 2017
\talkitem{The NANOGrav pulsar signal simulator}{International Pulsar Timing Array}{S\`{e}vres, France, July, 2017}
\talkitem{Late-time quadrupolar gravitational wave power in de Sitter space}{American Physical Society April Meeting}{Washington, DC, January, 2017}
\talkitem{Null Stream Approach with PTAs: Noise Characterization and Excess Power}{American Astronomical Society 227th Meeting}{Grapevine, Texas, January, 2017}
\talkitem{Assessing the null stream approach for source localization in PTAs}{Fall NANOGrav Meeting}{Urbana, Illinois, October, 2016}
\talkitem{Comparing transverse-traceless decompositions of symmetric tensors}{Int. Soc. for General Relativity and Gravitation 21st Meeting}{New York City, New York, July, 2016}
\talkitem{Null Stream Approach for finding Sky Position of Pulsar Timing Array sources}{American Physical Society April Meeting}{Salt Lake City, Utah, April, 2016}
\talkitem{A Cartan Geometry approach to the AdS/CFT}{Midwest Gravity Meeting}{Evanston, Illinois, October, 2015}
\talkitem{Tracing the AdS/CFT Degrees of Freedom using Cartan Geometry}{American Physical Society April Meeting}{Baltimore, Maryland, April 2015}
\talkitem{Conformal gravity, dark matter and time}{Midwest Gravity Meeting}{Rochester, MI, November, 2014}
\talkitem{Conformal gravity, dark matter and time}{APS Four Corners Meeting}{Orem, Utah, October 2014}
\talkitem{Time from the conformal symmetries of a Euclidean space}{Midwest Gravity Meeting} Milwaukee, Wisconsin, October 2013
\talkitem{Lorentzian geometry from the conformal symmetries of a Euclidean space}{Loops 13: International Conference on Quantum Gravity}{Waterloo, Canada, July 2013}
\talkitem{Testing Bimetric and Massive Gravity Theories using Multi-Messenger Astronomy}{GR20/AMALDI$\;$10}{Warsaw, Poland, July 2013}
\talkitem{Lorentzian spin connection from the conformal symmetries of a Euclidean space}{53rd Cracow School of Theoretical Physics}{Zakopane, Poland, June 2013}
\talkitem{General relativity in signature changing phase space}{Pacific Coast Gravity Meeting}{Davis, California, March 2013}
\talkitem{General relativity in phase space with a natural notion of time}{Pacific Coast Gravity Meeting}{Santa Barbara, California, March 2012}
\talkitem{A systematic construction of curved phase space: A gravitational gauge theory with symplectic form}{Loops 11: International Conference on Quantum Gravity}{Madrid, Spain, May 2011}
\talkitem{Quantum gravity in relativistic phase space}{Intermountain Graduate Research Symposium}{Logan, Utah, March 2010}
\talkitem{Multiple Spherical Shells in Schwarzschild Spacetime}{12th Marcel Grossman Gravity Meeting}{Paris, France, July 2009}
\talkitem{Single Spherical Shells in Schwarzschild Spacetime}{Pacific Coast Gravity Meeting}{Eugene, Oregon, March 2009}


\end{etaremune}
%\vspace{-0.6cm}
\subsection{Posters}
\begin{etaremune}[leftmargin=8mm]
\small
\talkitem{Implementing parameterized equation of state models in BILBY}{Ligo-Virgo-Kagra Collaboration Meeting}{Evanston, Illinois March 2023}
\talkitem{Contraining the Neutron Star Equation of State}{Kenyon College Summer Science Fair}{Gambier, Ohio, September, 2021}
\end{etaremune}

\begin{comment}
\newpage
\section{References}


	\renewcommand\labelitemi{\blucirc}
	\begin{itemize}[leftmargin=8mm]
	\setlength\itemsep{0.05mm}
	\item \textbf{Dr. Jeffrey Hazboun} \emph{NANOGrav Noise Budget Group Chair/ PhD Advisor}
       			\newline Professor of Physics
		\newline Oregon State University, Corvallis, OR 97331
		\newline \qquad email:\url{jeffrey.hazboun@oregonstate.edu}
		\newline
	\item \textbf{Dr. Xavier Siemens} \emph{NANOGrav PI}
        		\newline Professor of Physics
		\newline Oregon State University, Corvallis, OR 97331
		\newline \emph{Phone:} 541-737-7512 \qquad \emph{email:} \url{xavier.siemens@oregonstate.edu}
		\newline
	\item \textbf{Dr. Leslie Wade} \emph{Undergraduate Advisor}
       		\newline Professor of Physics
		\newline Kenyon College, Gambier, OH 43022
		\newline \emph{Phone:} 740-427-5356 \qquad email:\url{wadel@kenyon.edu}
		\newline
	\item \textbf{Dr. Madeline Wade} %\emph{Post Doctoral Advisor}
       		\newline Professor of Physics
		\newline Kenyon College, Gambier, OH 43022
		\newline \emph{Phone:} 740-427-5268 \qquad email:\url{wadem@kenyon.edu}
		\newline

	\end{itemize}
%\subsection{Teaching}
\end{comment}

\end{document}
